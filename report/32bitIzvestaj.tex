\documentclass[a4paper,12pt, projekat]{etf}

\usepackage[intlimits]{amsmath}
\usepackage{amsmath, amsfonts, amssymb, graphicx}

\usepackage[serbian]{babel}
\usepackage[T1]{fontenc}
\usepackage[utf8]{inputenc}
\usepackage{graphicx}

\addto\captionsserbian{\renewcommand{\bibname}{Literatura}}

\title{Poredjenje vremena filtriranja na STM32F4 mikrokontroleru i PC-u}
\author{Lazar Caković}
\indeks{3083/2016}
\date{jul 2018.}
\mentor{prof.~dr Nenad Jovi\v{c}i\'{c}}
\predmet{32bitni mikrokontroleri}

\begin{document}

	\maketitle

	\tableofcontents

	\listoffigures

	\newpage

	\chapter{Uvod}
        Cilj projekta iz predmeta Ma\v{s}inska vizija, na master studijama odseka
        za Elektroniku, bio je da se na tastaturi koja se koristi u industrijske
        svrhe detektiju stanja u kojima se nalazi. Prepoznavanje stanja je
        uradjeno kori\v{s}\'{c}enjem kamere koja je javno dostupna. Testiranje
        rada celog sistema je uradjeno promenom stanja tastature, ispisom
        prepoznatog stanja, i poredjenjem rezultata algoritma sa realnim
        stanjem tastature. Realizacija projekta je odradjena tako
        da se lako moze iskoristiti u sklopu ve\'{c}eg sistema, tako da informacije
        prikupljene uz pomo\'{c} ovog programa mogu biti iskori\v{s}\'{c}ene u nekoj daljoj
        obradi.

        \newpage

        \chapter{Opis sistema}

        \newpage

        \chapter{Softverska implementacija}

        \section{Deo za procesiranje}

        \chapter{Apendix}

\begin{verbatim}


\end{verbatim}

\end{document}
