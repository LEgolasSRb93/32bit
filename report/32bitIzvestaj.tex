\documentclass[a4paper,12pt, projekat]{etf}

\usepackage[intlimits]{amsmath}
\usepackage{amsmath, amsfonts, amssymb, graphicx}

\usepackage[serbian]{babel}
\usepackage[T1]{fontenc}
\usepackage[utf8]{inputenc}
\usepackage{graphicx}

\addto\captionsserbian{\renewcommand{\bibname}{Literatura}}

\title{Poredjenje vremena filtriranja na STM32F4 mikrokontroleru i PC-u}
\author{Lazar Caković}
\indeks{3083/2016}
\date{jul 2018.}
\mentor{prof.~dr Nenad Jovi\v{c}i\'{c}}
\predmet{32bitni mikrokontroleri}

\begin{document}

	\maketitle

	\tableofcontents

	\listoffigures

	\newpage

	\chapter{Uvod}
        Cilj projekta iz predmeta 32-bitni mikrokontroleri, na master studijama odseka
        za Elektroniku, bio je da se uporedi filtriranje signala u vremenskom i frekvencijskom
		domenu na mikrokontroleru STM32F407 i PC-u kori\v{s}\'{c}enjem MATLAB programskog paketa.
		Poredjenje filtriranja se vr\v{s}ilo na brzinu izvr\v{s}avanja filtriranja, kao i na 
		odstupanje signala dobijenog filtriranjem na mikrokontroleru od onog dobijenog u programskom
		paketu MATLAB. Projekat je radjen u softverskim paketima MATLAB i IAR Embedded Workbench for ARM,
		a na plo\v{c}i STM32F4 Discovery. 

        \newpage

        \chapter{Opis sistema}
		U implementaciji projekta, sistem je konfigurisan tako da se deo projekta izvr\v{s}ava na PC-u,
		u okviru MATLAB programskog paketa. Dok se drugi deo izvr\v{s}ava na samoj razvojnoj plo\v{c}i.
		Na PC-u je izgenerisan signal koji je potrebno obraditi. Razvojna plo\v{c}a Discovery je povezana
		na PC, i kori\v{s}\'{c}enjem \texit{semihosting-a} implementirana je emulacija \textit{file system-a}
		tako da je omogu\'{c}eno \v{c}itanje i upis u fajlove PC-a.

        \newpage

        \chapter{Softverska implementacija}
		Kao \v{s}to je ve\'{c} navedeno, softverska implementacija je izvr\v{s}ena na PC-u, u okviru MATLAB
		programskog paketa, i na strani mikrokontrolera u okviru IAR razvojnog okru\v{z}enja. Pa \'{c}e tim
		redom i biti opisane.
		
		\section{MATLAB implementacija}
		U ovom delu softverske implementacije, izgenerisan je signal kao zbir dve sinusoide razli\v{c}itih
		frekvencija koji je potrebno izfiltrirati. Konkretno, kao primer, uzet je zbir dve sinusoide od 2KHz
		i 10KHz, sa frekvencijom odabiranja od 50KHz, u 256 ta\v{c}aka. Ovaj broj ta\v{c}aka je uzet zbog 
		ograni\v{c}enja razvojne plo\v{c}e koja se koristi u drugom delu implementacije, po\v{s}to se ovde 
		izgenerisan signal koristi u daljoj implementaciji. Takodje, u ovom delu se generi\v{s}u i odbirci
		filtra koji \'{c}e biti kori\v{s}\'{c}en u filtriranju kako u ovom delu, tako i u drugom delu 
		implementacije. Filtar koji se koristi je filtar propusnik niskih u\v{c}estanosti, 120og reda, sa 
		grani\v{c}nom u\v{c}estanosti od 8kHz, i slabljenjem u propusnom opsegu od 0.01dB i u nepropusnom 
		opsegu od 80dB. Za dobijanje odbiraka filtra koristi se ugradjena funkcija u MATLAB-u \textbf{fircegrip}.
		
		%ubaciti slike signala.
		
		Filtriranje signala u MATLAB-u se vr\v{s}i u vremenskom i frekvencijskom domenu. Filtriranje signala
		u frekvencijskom domenu se vr\v{s}i kori\v{s}\'{c}enjem ugradjene funkcije \textit{filter}, uz pomo\'{c}
		koje se dobija isfiltriran signal kao na slici ispod.
		
		%ubaciti slike izfiltriranog singala
		
		Filtriranje u frekvencijskom domenu vr\v{s}i se mno\v{z}enjem FFT predstave signala i filtra. Nakon 
		\v{c}ega se dobija isti rezultat kao u prethodnoj implementaciji.
		
		%ubaciti slike izfiltriranog signala
		
		Kao \v{s}to je navedeno, odbirci signala koji su generisani, upisuju se u binarne fajlove koji \'{c}edef
		se koristiti u implementaciji na mikrokontroleru.
		
        \section{Implementacija na mikrokontroleru}
		U ovom delu softverske implementacije, filtriranje ve\'{c} izgenerisanih signala se vr\v{s}i na 
		strani mikrokontrolera. Ova implementacija je izvr\v{s}ena u IAR Embedded Workbench for ARM razvojnom
		okru\v{z}enju koji sadr\v{z}i IAR kompajler za ARM arhitekture.
		
		Kako su odbirci signala koji je potrebno izfiltrirati, kao i odbirci filtra koji se koristi za filtriranje 
		sme\v{s}teni u binarne fajlove, na strani mikrokontrolera kori\v{s}en je mehanizam \textit{semihosting-a}.
		\textit{Semihosting} je mehanizam koji omogu\'{c}ava kodu koji se pokre\'{c}e na ARM platfomi da komunicira 
		i koristi ulazno-izlazne mehanizme na \textbf{host} PC-u, odnosno na PC-u na kome se pokre\'{c}e razvojno 
		okru\v{z}enje. Ovaj mehanizam omogu\'{c}ava sve operacije sa fajlovima na ARM mikrokontroleru, kao i ispis 
		i \v{c}itanje u Debug terminalu unutar razvojnog okru\v{z}enja. Kori\v{s}enjem ovog mehanizma, pojednostavljen 
		je pristup podacima koji se nalaze na samom PC-u, ali ovaj mehanizam ima ograni\v{c}enje, a to je da se mo\v{z}e 
		koristiti samo u delu projektovanja i debug-ovanja sistema. Odnosno, samo ukoliko se debugger koristi. Ukoliko 
		se \v{z}eli posti\'{c}i da sistem radi samostalno, potrebno je na\'{c}i novi na\v{c}in za dobijanje podataka na 
		mikrokontroleru, bilo to prenosom preko serijskog interfejsa, ili na neki drugi na\v{c}in. 
		Nakon dobijanja svih potrebnih podataka na mikrokontroleru, implementirana su dva na\v{c}ina filtriranja, kao i 
		u prethodnom delu. A to je filtriranje u vremenskom i frekvencijskom domenu. Oba na\v{c}ina filtriranja su 
		implementirana kori\v{s}\'{c}enjem CMSIS bibilioteke i njenih fukncija za \textbf{fir} filtriranje, kao i 
		funkcija za \textbf{fft}.
		Za filtriranje u vremenskom domenu, kori\v{s}\'{c}ena je funkcija \textbf{arm_fir_f32} koja vr\v{s}i filtriranje 
		signala \textbf{fir} filtrom. (dopisati jos o ovome). Nakon \v{c}ega se izfiltriran signal sme\v{s}ta u izlazni 
		binarni fajl. Nakon \v{c}ega se mo\v{z}e porediti u MATLAB-u.
		Takodje, filtiranje u frekvencijskom domenu se vr\v{s}i kori\v{s}\'{c}enjem funkcija iz CMSIS biblioteke, i to 
		konkretno funkcije \textbf{arm_cfft_f32}, kojom se dobija FFT ulaznog signala i filtra. Nakon \v{c}ega se vr\v{s}i 
		mno\v{z}enje FFT predstava signala i filtra u kompleksnom domenu. I na posletku, istom funkcijom \textbf{arm_cfft_f32} 
		se radi inverzna FFT. Kao i u prethodnom delu, izlazni signal se sme\v{s}ta u binarni fajl, nakon \v{c}ega se mo\v{z}e 
		porediti u MATLAB-u.
		
		% u kodu postoji deo za merenje performansi, samo pokrenuti, i ubaciti skrin sot u izvestaj

        \chapter{Apendix}

\begin{verbatim}


\end{verbatim}

\end{document}
